\documentclass{article}
\usepackage{graphicx}
\usepackage{titletoc}
\usepackage{titlesec}
\usepackage{geometry} 
\usepackage{fontspec, xunicode, xltxtra}
\usepackage{float}
\usepackage{cite}
\usepackage{amsmath}
\usepackage{listings}
\usepackage{titletoc}
\usepackage{booktabs}

\geometry{left=3cm,right=3cm,top=3cm,bottom=3cm}
\DeclareMathOperator*{\argmin}{argmin}
\DeclareMathOperator*{\argmax}{argmax}
\DeclareMathOperator*{\logit}{logit}
\DeclareMathOperator*{\var}{var}
\DeclareMathOperator*{\cov}{cov}
\DeclareMathOperator*{\expec}{E}
\DeclareMathOperator*{\deriv}{d}
\DeclareMathOperator*{\const}{constant}

\begin{document}
\title{\textsf{Homework 9 for Bayesian Data Analysis}}
\author{Fan JIN\quad (2015011506)}
\maketitle

\section*{Question 14.3}
{
    Assuming uniform prior distribution for $\beta|\sigma$, we have
    $$p(\beta|\sigma, y) \propto p(y | \beta, \sigma) p(\beta | \sigma)$$
    $$\propto p(y | \beta, \sigma) \propto \exp{\left( -\frac{1}{2\sigma^2} (y-X\beta)^T (y-X\beta) \right)}.$$

    Note the fact that $$(y-X\beta)^T (y-X\beta) = (\beta-\hat{\beta})^T X^T X (\beta-\hat{\beta}) + \const,$$ we have
    $$p(\beta|\sigma, y) \propto \exp{\left( -\frac{1}{2\sigma^2} (y-X\beta)^T (y-X\beta) \right)}$$
    $$\propto \exp{\left( -\frac{1}{2\sigma^2} (\beta-\hat{\beta})^T X^T X (\beta-\hat{\beta}) \right)}$$
    $$= \exp{\left( -\frac{1}{2} (\beta-\hat{\beta})^T ((X^T X)^{-1} \sigma^2)^{-1} (\beta-\hat{\beta}) \right)},$$
    which implies that $$\beta | \sigma, y \sim N(\hat{\beta}, (X^T X)^{-1} \sigma^2).$$
}

\section*{Question 14.4}
{
    Assuming the noninformative prior $p(\beta, \log{\sigma}) \propto 1$, or $p(\beta, \sigma^2) \propto \sigma^{-2}$, we obtain
    $$p(\sigma | y) = p(\beta, \sigma^2 | y) / p(\beta | \sigma^2, y) \propto p(\beta, \sigma^2) p(y | \beta, \sigma^2) / p(\beta | \sigma^2, y)$$
    $$\propto \frac{\sigma^{-2} \cdot \sigma^{-n} \exp{\left( -\frac{1}{2\sigma^2} (y-X\beta)^T (y-X\beta) \right)}}{[\det{((X^T X)^{-1} \sigma^2)]^{-1/2} \cdot \exp{\left( -\frac{1}{2\sigma^2} (\beta-\hat{\beta})^T X^T X (\beta-\hat{\beta}) \right)}}}$$

    Fix $\beta = \hat{\beta}$ in the formula above, and it follows that
    $$p(\sigma | y) \propto \frac{\sigma^{-2} \cdot \sigma^{-n} \exp{\left( -\frac{1}{2\sigma^2} (y-X\hat{\beta})^T (y-X\hat{\beta}) \right)}}{[\det{((X^T X)^{-1} \sigma^2)]^{-1/2}}}$$
    $$= \frac{\sigma^{-2} \cdot \sigma^{-n} \exp{\left( -\frac{1}{2\sigma^2} (y-X\hat{\beta})^T (y-X\hat{\beta}) \right)}}{[\sigma^{2k} \cdot \det{((X^T X)^{-1})]^{-1/2}}}$$
    $$\propto \sigma^{-2-n+k} \cdot \exp{\left( -\frac{1}{2\sigma^2} (y-X\hat{\beta})^T (y-X\hat{\beta}) \right)}.$$

    Compare this expression to the Inverse-$\chi^2$ distribution\footnote{https://en.wikipedia.org/wiki/Inverse-chi-squared\_distribution}, we find that 
    $$s^2 = \frac{1}{n-k} (y-X\hat{\beta})^T (y-X\hat{\beta}).$$
}

\section*{Question 14.7}
{
    $\widetilde{y}$ conforms a normal distribution, as $\widetilde{y}$ is a linear combination of $\beta$, and $p(\beta | \sigma, y)$ is normal.
}

\section*{Longley data 1}
{
    The mean of Inverse-$\mathrm{\chi^2}(n-k, s^2)$ is\footnote{https://en.wikipedia.org/wiki/Scaled\_inverse\_chi-squared\_distribution} 
    $$\frac{(n-k)s^2}{n-k-2}.$$
    The calculated result is 
    \begin{lstlisting}
        [1] 6.003243
    \end{lstlisting}
}

\section*{Longley data 2}
{
    The posterior mean of $\beta$ under a conjugate prior $(\beta_0, \beta_1) \sim N(0, I)$ is
    \begin{lstlisting}
    (Intercept)           X 
     -76.759663    1.519031 
    \end{lstlisting}
}

\section*{Longley data 3}
{
    \begin{lstlisting}
    > mod.full = BayesReg(dat$GNP.deflator, dat[, -1], g=nrow(dat))

    PostMean PostStError Log10bf EvidAgaH0
    Intercept 101.6813      0.7431                  
    x1         23.8697     25.1230 -0.3966          
    x2          3.1068      6.6053 -0.5603          
    x3          0.7078      2.5134 -0.5954          
    x4        -11.0111     10.9543 -0.3714          
    x5         -6.1556     32.7640 -0.6064          
    x6          0.7402     10.7025  -0.614          


    Posterior Mean of Sigma2: 8.8342
    Posterior StError of Sigma2: 13.0037

    > mod = BayesReg(dat$GNP.deflator, dat[, c(-1, -3)], g=nrow(dat))

    PostMean PostStError Log10bf EvidAgaH0
    Intercept 101.6813      0.7494                  
    x1         14.6884     15.9493 -0.4094          
    x2         -0.3300      1.2139 -0.5968          
    x3         -9.9863     10.8264 -0.4087          
    x4          8.6301      9.3120 -0.4068          
    x5         -3.5398      5.6814 -0.5194          


    Posterior Mean of Sigma2: 8.9846
    Posterior StError of Sigma2: 13.2249

    > bf.full / bf
    Bayes factor analysis
    --------------
    [1] GNP + Unemployed + Armed.Forces + Population + Year + Employed : 0.1443632 ±0%

    Against denominator:
    GNP.deflator ~ GNP + Armed.Forces + Population + Year + Employed 
    ---
    Bayes factor type: BFlinearModel, JZS
    \end{lstlisting}

    We find that the reduced model has similar coefficients to the full model. And the ratio of Bayes factor is much less than 1, which means the reduced model is more likely. 
}

\clearpage
\end{document}
